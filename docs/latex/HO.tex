One of the cornerstones of physics\textquotesingle{} approximations is the harmonic oscillator. The energy of an $N$-\/dimensional harmonic oscillator is given by\+: \[ E(\vec{q};\vec{k})=\frac{1}{2}\vec{k}\cdot(\vec{q}\circ\vec{q}), \] where $\vec{q}$ is a vector of $N$ coordinates and $\vec{k}$ is a vector of $N$ force constants such that the $q_i$ has force constant $k_i$. $\circ$ denotes the Hadamard product (element-\/wise multiplication), i.\+e. $\left[\vec{q}\circ\vec{q}\right]_{i}=q_i^2$.

The derivative w.\+r.\+t. to $\vec{q}$ is\+: \[ \left[\frac{\partial E(\vec{q};\vec{k})}{\partial\vec{q}}\right]_i=k_i q_i, \] the Hessian is just $\vec{k}$ down the diagonal. All higher derivatives are 0. Doesn\textquotesingle{}t get much easier than that.

\subsection*{Notes on H.\+O. as Implemented in Force\+Man\+II}


\begin{DoxyItemize}
\item As with all force field quantities we assume atomic units are used throughout
\item We explicitly include the 1/2 in our energy formula
\item Usually a H.\+O. measures the energy penelty caused by displacemet from some value $\vec{q_0}$. To simplify the underlying code we assume that input coordinates $\vec{q}=\vec{q}^\prime-\vec{q_0}$ where $\vec{q}^\prime$ is the actual value of the bond, angle, etc. 
\end{DoxyItemize}