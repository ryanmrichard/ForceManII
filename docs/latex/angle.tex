The angle class is responsible for normal angle bending terms (i.\+e. those among three bonded atoms). Assuming the vectors $\vec{r_{12}}$ and $\vec{r_{32}}$ are vectors parallel to the bond between atoms 1 and 2 and the bond between atoms 2 and 3 respectively the cosine of the angle between these vectors, $\cos\theta$ is given by\+:

\[ \cos\theta=\frac{\vec{r_{12}}\cdot\vec{r_{32}}} {r_{12} r_{32}}, \] where $r$ denotes the magnitude of $\vec{r}$. We can also express this in terms of the $\sin\theta$\+: \[ \sin\theta=\frac{\left(\vec{r_{12}}\times\vec{r_{32}}\right)\cdot\vec{n}} {r_{12} r_{32} n}, \] where $\frac{\vec{n}}{n}$ is a unit vector perpendicular to both $\vec{r_{12}}$ and $\vec{r_{32}}$. Consequentially, we can also use the tangent of the angle\+: \[ \tan\theta=\frac{\left(\vec{r_{12}}\times\vec{r_{32}}\right)\cdot\vec{n}} {\vec{r_{12}}\cdot\vec{r_{32}}n}. \] 